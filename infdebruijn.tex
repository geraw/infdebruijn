\documentclass{article}
%\documentclass[letterpaper, 10 pt, conference]{ieeeconf}  

%\IEEEoverridecommandlockouts        % This command is only needed if you want to use the \thanks command
%\overrideIEEEmargins                        % See the \addtolength command later in the file to balance the column lengths


% Name clashes workaround
%\makeatletter
%\let\IEEEproof\proof
%\let\IEEEendproof\endproo f
%\let\proof\@undefined
%\let\endproof\@undefined
%\makeatother




\usepackage{epsfig} 
\usepackage{amsmath} 
\usepackage{amssymb} 
\usepackage{amsthm}  
\usepackage{hyperref}
\usepackage{tikz}
\usetikzlibrary{automata,arrows,decorations.pathmorphing,decorations.markings,positioning}
\usetikzlibrary{shapes}
%\usepackage{balance}
\usepackage{scrextend}
\addtokomafont{labelinglabel}{\sffamily}


% Theorem like env.
\newtheorem{theorem}{Theorem}
\newtheorem{proposition}[theorem]{Proposition}
\newtheorem{claim}[theorem]{Claim}
\newtheorem{lemma}[theorem]{Lemma}

\theoremstyle{definition}
\newtheorem{definition}[theorem]{Definition}     
\newtheorem{conjecture}[theorem]{Conjecture}
\newtheorem{observation}[theorem]{Observation}
\newtheorem{example}[theorem]{Example}
\newtheorem{remark}[theorem]{Remark}
\newtheorem{corrolary}[theorem]{Corrolary}



%\newcommand{\RotateLeft}{rlft}
\DeclareMathOperator{\RotateLeft}{RotLft}



%------------------------------------------- preamble ---------------------------

\newcommand{\R}{{\mathbb{R}}}
\newcommand{\N}{{\mathbb{N}}}
\newcommand{\Z}{{\mathbb{Z}}}

\newcommand{\REF}[2]{#1~\ref{#2}}

\newcommand{\T}[1]{\langle{#1}\rangle}
\newcommand{\rr}[2]{R^{#2}({#1})}
\newcommand{\rl}[2]{R^{-{#2}}({#1})}


\newcommand{\AorB}[2]{\bigl(#2\text{?}A(#1)\text{:}B(#1)\bigr)}


\pagestyle{plain}


\title{An Infinite de Bruijn Sequence}

\author{}




%%%%%%%%%%%%%%%%%%%%%%%%%%%%%%%%%%%%%%%%%%%%%%%%%%%%%%%%%%%%%%%%%%%%%%%%%%%%%%%%%%%%%%%%%%%%%%%%%%%%

\begin{document}
\maketitle

\section{A cycle joining Construction}


\begin{definition}
	A word in $\N^*$ is called a \emph{key word} if it is bigger in right-to-left lexicographical order than all of its rotations.
\end{definition}

\begin{definition}[Cycle Construction]
For $n\in \N$, let $k_1,k_2,\dots$ be an enumeration in right-to-left lexicographic order of all the key words of length $n$ and let $z(k_i)$ be the number of leading zeros in $k_i$. Define $C_0=\T{0^n}$ and, for $i>0$, let $C_i$ be the sequence obtained from $C_{i-1}$ by adding 
	$$\T{\RotateLeft(k_i,z(k_i)+1),\RotateLeft(k_i,z(k_i)+2),\dots,\RotateLeft(k_i,z(k_i))}$$ 
after the word $(\sigma-1)w$ where $\sigma w=\RotateLeft(k_i,z(k_i))$. \footnote{This is always possible because  $(\sigma-1)w$ is a rotation of a key word that is smaller than $k_i$ in left-to-right lexicographic order, i.e., it came earlier in the enumeration.}  
\end{definition}


\begin{proposition}
For every $l>0$ there is some $i>0$ such that the first $l$ words in $C_i$ are also the first $l$ words in $C_j$ for any $j>i$.
\end{proposition}

\begin{definition}
Let $C_\infty$ be the infinite sequence whose prefixes are all prefixes of an infinite number of elements in $\{C_0,C_1,\dots\}$.
\end{definition}

\begin{proposition}
The sequence $C_\infty$ is a de Bruijn sequence of order $n$, i.e., it is a Hamiltonian path over the infinite de Bruijn graph of order $n$ whose vertexes are the words $\N^n$ and whose edges are $\{ \T{ \sigma w, w \sigma'} \colon \sigma, \sigma' \in \N, w \in \N^{n-1} \}$.
\end{proposition}


\begin{definition}[Last on Necklace]
Let $last(w)$ be true if and only if $w=\RotateLeft(k_i,z(k_i))$ for some $i$. i.e., if $w$ is the last member of its necklace (set of rotations) in the sequence. 
\end{definition}


\begin{definition}
Let $nxt\colon \N^n \to \N^n$ be defined by
$$nxt(\sigma w)=\begin{cases} 
w(\sigma+1) & \text{if }last((\sigma+1)w) \\
w0          & \text{if }last(\sigma w) \\
w\sigma     & \text{otherwise} 
\end{cases}$$
\end{definition}

\begin{proposition}
For every $w\in \N^n$, $nxt(w)$ is the word that follows $w$ in the sequence $C_\infty$.
\end{proposition}


\begin{definition}
Let $prv\colon \N^n \to \N^n$ be defined by
$$prv(w\sigma)=\begin{cases} 
(\sigma-1)w & \text{if }last(\sigma w) \\
\sigma'w   & \text{if $\sigma =0$  and $\sigma'$ is the maximal such that $last(\sigma' w)$} \\
\sigma w     & \text{otherwise} 
\end{cases}$$
\end{definition}
 
\begin{proposition}
For every $w\in \N^n\setminus \{0^n\}$, $prv(w)$ is the word that precedes $w$ in the sequence $C_\infty$.
\end{proposition}

\begin{proposition}
Both $prv$ and $nxt$ can be computed in $O(n^2)$ time and space.
\end{proposition}   

\begin{proposition}[Prefer Maximum]
For every $\sigma \in \N$ and $w \in \N^{n-1}$, the word $\sigma w$ comes before the word $(\sigma+1) w$ in $C_\infty$.
\end{proposition}
 


	
\section{Forward and backwards transformations}

\begin{definition}
	For a parameter $n$, Let $L \subset \N^+$ be the set of non-periodic words over the alphabet $\Sigma=\N$ that are bigger in Arabic (right-to-left) lexicographical order than all of their rotations. Let $L_n$ be the set of all the words in $L$ whose length divides $n$.
\end{definition}


\begin{definition}
	For a word $w=w_1\cdots w_{n-1}w_n$ let  $\rr{w}{}= w_{n}w_1\cdots w_{n-1}$ be the rotation of $w$ to the right. Then, the nested invocation $\rr{w}{m}$ is the $m$ letter rotation to the right and its inverse $\rl{w}{m}$ is the $m$ letter rotation to the left.
\end{definition}


\begin{definition}\label{forward}
For a word $w$ whose length is smaller or equal than $n$, let  $f(w)$ be the transformation defined by successive applications of the following steps to $w$:
\begin{enumerate}
	\item[$f_1$:] Increase the first letter of the word by one. 
	\item[$f_2$:] Pad with zeros on the left to get a word of length $n$.
	\item[$f_3$:] Apply the substitution rules $u(vu)^+ \mapsto vu$ and then $w^+ \mapsto w$, with the longest possible $u$ and the shortest possible $w$.
\end{enumerate}
\end{definition}

\begin{definition}\label{backward}
For a a word $w$ whose length is smaller or equal than $n$, let  $b(w)$ be the transformation defined by successive applications of the following steps to $w$:
\begin{enumerate}
	\item[$b_1$:] Expand $w$ to $u w^m$ where $m=\lfloor n/ |w| \rfloor$ and $u$ is the suffix of length $n-m|w|$ of $w$.
	
	\item[$b_2$:] Remove leading zeros.
	\item[$b_3$:] Decrease the first letter by one.
\end{enumerate}
\end{definition}

\begin{observation}
	For any $w\in L_n$, $f(b(w))=b(f(w))=w$.
\end{observation}



\begin{proposition}
	If we start with $w(0)=0$ and generate a sequence of words by $w({i+1})=f(w(i))$, we get an enumeration of all the words in $L$ whose length is smaller or equal to $n$.
\end{proposition}
\begin{proof}
	This is a version of Duval's algorithm with a reversed order of the alphabet and a reversed order of letters in a word. 
\end{proof}

\begin{definition} \label{star-functions}
	Let $f^*(w)$ be the first word in ${f(w),f(f(w)),\dots}$ whose length divides $n$ and, similarly, let $b^*(w)$ be the first word in ${b(w),b(b(w)),\dots}$ whose length divides $n$. 
\end{definition}


\begin{definition}
	Let $w(0),w(1),...$ be the sequence generated by starting with $w(0)=0$ and then continuing ad infinitum by $w{(i+1)}=f^*(w(i))$ and let $w^\infty\in \N^\omega$ be the concatenation of all these words.
\end{definition}



\section{Where can I find $w$ as a sub-word of $w^\infty$?}

In this section we point at the position of an arbitrary word $w$ as a sub-word of  $w^\infty$ relative to the position of the a corresponding word in $L_n$. This is given in Proposition~\ref{simple-pos} and in Proposition~\ref{advanced-pos}. Towards the proofs of these propositions, we first establish some technical results about the functions $b$ and $b^*$ specified, respectively, in Definition~\ref{backward} and in Definition~\ref{star-functions}.


\begin{proposition}
	\label{w-at-the-tail}
	If $w \in L_n$ and $|w| \neq n$ then  $b(w)=uw$ for some non-empty word $u$. 
\end{proposition}
\begin{proof}
	The first transformation $b_1$ extends $w$ to the left producing the word $b_1(w)=uw^m$ where $u$ is a tail of $w$. Since $w \in L_n$ and because it contains a letter $\sigma$ that is not zero, we have, by maximality of $w$ among its rotations in right-to-left lexicographical order,  that its last letter is not zero. The last letter of $u$ is the last letter of $w$ so it is also not zero. This gives us that the next transformation $b_2$, that deletes trailing zeros, leaves at least the last copy of $w$ and the last letter of the before-last (full or partial) copy at the tail of $b_1(w)$. Thus, $b_2(b_1(w))=uw$ where $u$ is a non-empty word whose first letter is not zero. Then, the last transformation $b_3$ only decreases the first letter of $u$ by  one which gives us that $b(w)=b_3(b_2(b_1(w)))=vw$ for some non-empty word $v$.	
\end{proof}
	
\begin{proposition}
	\label{hat-w-at-the-tail}
	For any $w=0^l \sigma \hat{w}\in L_n$ where $\sigma$ is a non-zero letter there is a non-empty word $u$ such that $b(w)=u\hat{w}$.
\end{proposition}
\begin{proof}
	If $|w|\neq n$ the proof follows by Proposition~\ref{w-at-the-tail}. If $|w|=n$ then $b_1(w)=w$, $b_2(b_1(w))=\sigma\hat{w}$, and $b_3(b_2(b_1(w)))=(\sigma-1)\hat{w}$ and the claim follows as well.
\end{proof}
	
	
\begin{proposition}\label{simple-pos}
	Let $w$ be an arbitrary word in $\N^n$ and let and let $\bar{w}=f_3(w)$. Let $l$ be the (possibly zero) number of trailing zeros (from the left) in $w$.  Then, for all $0 \leq i \leq |w|-l-1$, the word $\rr{\bar{w}}{i}$ comes $i+n-|w|$ letters before $w$ as a sub-word of $w^\infty$.
\end{proposition}



\begin{proposition}\label{simple-pos}
	For a given $w \in L_n$, let $l$ be the number of trailing zeros (from the left) in $w$ and let $\bar{w}=b_1(w)$. Then, for all $0 \leq i \leq |w|-l-1$, the word $\rr{\bar{w}}{i}$ comes $i+n-|w|$ letters before $w$ as a sub-word of $w^\infty$.
\end{proposition}
\begin{proof}
	By Proposition~\ref{tail-is-preseved*} the words that come before $w$ ends with the last $|w|-l$ letters of $w$. In particular, the $n$ letter word that starts $i+n-|w|$ before $w$ is $\rr{\bar{w}}{i}$.
\end{proof}

\begin{proposition}\label{advanced-pos}
	For a given $w \in L_n$, let $l$ be the number of trailing zeros (from the left) in $w$ and let $\bar{w}=b_1(w)$. Then, for all $|w|-l \leq i \leq n-1$ the word $\rr{\bar{w}}{i}$ comes $i-(n-|f_3(u)| \pmod n)$ letters before the first $u \in \T{0^{m-1} (\bar{w}_{m} + 1) \bar{w}_{m+1}\cdots \bar{w}_n}_{m=i+1}^{n}$ that is in $L_n$. 
\end{proposition}


\begin{proposition}
	The word $w^\infty$ contains all the words in $\N^n$ as subwords.
\end{proposition}
\begin{proof}
	Any word of length $n$ is a rotation of the expansion of a word in $L_n$.
\end{proof}


\begin{proposition}
	For any $k$ the prefix $w^\infty_1 \cdots w^\infty_{k^n}$ is an $n$-order de Bruijn sequences. Moreover, it is the reversed of the n-order prefer-max sequence on the alphabet $\T{0,\dots,k-1}$ (in this order). 
\end{proposition}
\begin{proof}
	Counting argument + arguing that if $|w| =n-1$ and $\sigma_1 < \sigma_2$ then $w\sigma_1$ comes before $w\sigma_2$ as subwords of $w^\infty$.
\end{proof}


\begin{proposition}
For $w \in \N^n$, let $i$ be the minimal index such that $\rl{w}{i} \in L$ and let $\bar{w}=\rl{w}{i}$. Let $\bar{w}^+=\bar{w}_{1..i}(\bar{w}_{i+1}+1)\bar{w}_{(i+2)..n}$, i.e., the word obtained by increasing the $(i+1)th$ letter of $\bar{w}$ by one.  Then, the function
$$next(w)=
\begin{cases}
f^*(f_3(w))_1 & \text{if } w \in L; \\
w_1+1         & \text{if }\bar{w}_{1..i}= 0^i \wedge \bar{w}^+ \in L\wedge \max{(\bar{w}^+_{1..(n-1)})} \leq \max(w); \\
0             & \text{if }\bar{w}_{1..i}= 0^i \wedge (\bar{w}^+ \notin L \vee \max{(\bar{w}^+_{1..(n-1)})} > \max(w)); \\
w_1           & \text{otherwise.}
\end{cases}$$
represents the mapping of a word $w$ to the letter that follows the (one and only) occurrence of $w$ as a subword of $w^\infty$.
\end{proposition}

\begin{definition}
	Let $w(0)=0, w(1)=f^*(w(0)),\dots,w(i)=f^*(w{(i-1)}),\dots$ be our enumeration of all the words in $L_n$. Let $w^{(i)}=w(0)\cdots w(i)$ be the concatenation of the first $i$ words in this enumeration and let $u(j)=w^{(i)}_{j-n+1}\cdots w^{(i)}_j$ be the ``window'' of length $n$ before the $j$th letter in $w^{(i)}$.
\end{definition}

\begin{proposition}
	Let $w=w(i)$ for some $i$ and let $l$ be the number of leading zeros in $w$.
	Then, inserting the cycle $\T{R^{-l-n-1}(w),\dots,R^{-l}(w)}$ to $\T{u(j)}_{j=0}^{i-1}$
	after the word obtained from $R^{-l}(w)$ by decreasing its first letter by one
	yields the sequence $\T{u(j)}_{j=0}^{i}$.
\end{proposition}	




\section{Where can I find $w$ as a sub-word of $w^\infty$? (second try...)}


\begin{definition}
	For a word $w$, $max(w)$ is the maximal digit in $w$.
\end{definition}

\begin{definition}
	A word $u\in \mathbb{N}^n$ corresponds to $w\in L_n$ if $u$ is a rotation of $w^{\frac{n}{|w|}}$. Note that each $u\in \mathbb{N}^n$ corresponds to exactly one word $w\in L_n$.
\end{definition}

\begin{proposition}
	\label{newforward}
	If $w\in L_n$ and $|w|<n$, then $f^*(w)=f(w)=0^{n-|w|}x$ for some word $x$.
\end{proposition}
\begin{proof}
	Write $f_1(w)=x$, $f_2(x)=0^{n-|w|}x$. Since $w\in L_n$ and $|w|<n$, $n-|w|\geq \frac{n}{2}$. Moreover, the last digit in $x$ is not zero. Hecne, $f(w)=f_3(0^{n-|w|}x)=0^{n-|w|}x$. Since $|0^{n-|w|}x|=n$, we have $f(w)=f^*(w)=0^{n-|w|}x$.
\end{proof}

\begin{proposition}
	\label{1-back}
	Take $|w|<n$ so that $w=w'k$ where $0<k=max(w)$, then $b(w)=uw$ and $max(u)\leq max(w)$. 
\end{proposition}
\begin{proof}
	Write $w=w'k$. Thus, $b_1(w)=xk(w'k)^r$, $r>0$. $b_2(xk(w'k)^r)=y(w'k)^r$. $b_3(y(w'k)^r)=uw'k=uw$. It is easy to see that $\max{u}\leq k$.
\end{proof}

%\begin{proposition}
%Take $|w|<n$ so that $w=w'k$ where $0<k=max(w)$. Write $b_1(w)=xw^m$ where $x$ is a non-empty suffix of $w$ (possibly $x=w$). Then, $b(w)=uw^m$ and $max(u)\leq max(w)$. 
%\end{proposition}

%\begin{proof}
%Note that $b(w)=b_3(b_2(xw^m))=yw^m$. This holds since the last digit in $x$ is $k>0$. Clearly, $max(y)\leq max(w)$. If $|yw^m|=n$, then $b^*(w)=b(w)$ and we are done with $u=y$. Otherwise, apply the previous Proposition several times.
%\end{proof}

\begin{proposition}
	\label{*-back}
	If $w\in L_n$, $|w|^m=n$, $m>1$ and $w= 0^l\sigma\hat{w}$ such that $\sigma\neq 0$, then $b^*(w)=u\hat{w}w^{m-1}$ for some $u$.
\end{proposition}

\begin{proof}
	Since $w\in L_n$ and $w\neq 0$, $b(w)$ is defined and 
	$$b(w)=(\sigma-1)\hat{w}w^{m-1}.$$
	If $b(w)\in L_n$ we are done, and otherwise $|b(w)|<n$ and several invocations of the previous proposition provide the required.
\end{proof}


\begin{proposition}
	Assume that $u\in \mathbb{N}^n$ corresponds to $w\in L_n$ such that $|w|<n$. then, $u$ is a subword of $w^\infty$. 
\end{proposition}

\begin{proof}
	If $u=0^n$, then $u$ is a prefix of $w^\infty$ and we are done. Otherwise, $w=0^l\sigma\hat{w}$ where $\sigma\neq 0$. Take $m$ such that $|w|^m=n$. Note that $m>1$. By Propositions \ref{newforward} and  \ref{*-back}, $b^*(w)wf^*(w)=x\hat{w}w^{m-1}w0^{|w|}y$, which is also a subword of $w^\infty$. Hence, 
	$$\hat{w}(0^l\sigma\hat{w})^m0^l \text{ is a subword of } w^\infty.$$
	$u$ is a rotation of $w^m$ thus $u$ is a subword of $\hat{w}(0^l\sigma\hat{w})^m0^l$ which implies that $u$ is a subword of $w^\infty$.
\end{proof}


\begin{proposition}
	Assume that $u=yx\in\mathbb{N}^n$ corresponds to $w=xy\in L_n$ where $|w|=n$. If $x\neq 0^r$, then $u$ is a subword of $w^\infty$. 
\end{proposition}

\begin{proof}
	We show that $u=yx$ is a subword of $b^*(w)w$. Write $x=0^l\sigma z$ where $\sigma\neq 0$. Thus, since $|w|=n$, $b(w)=(\sigma-1)zy$. If $b(w)=b^*(w)$, then 
	$$b^*(w)w=(\sigma-1)zyx$$
	and we get that $u$ is a subowrd of $b^*(w)b(w)$. Otherwise, $|(\sigma-1)zy|$ does not divides $n$, and in particular, $|(\sigma-1)zy|<n$. By applying Proposition \ref{1-back} several times, we get that $b^*(w)=v(\sigma-1)zy$ for some $v$, and $u=yx$ is a subword of $b^*(w)w=v(\sigma-1)zyxyx$.
\end{proof}

\begin{lemma}
	\label{z1}
	Assume that $w=0^lv\in L_n$ and $|w|=n$. Write $w=0^lz_1\sigma z_2$ where $\sigma$ is the first digit in $v$ such that $0^{l+|z_1|}(\sigma+1)z_2$ is lexicographically maximal among its rotations. Take $k\in \N$ and a suffix of $(\sigma z_2)$, $u$ such that $|u(\sigma z_2)^{k+1}|=|z_1(\sigma z_2)|$. Then, $u(\sigma z_2)^{k+1}= z_1(\sigma z_2)$.
\end{lemma}

\begin{proof}
	Assume for a contradiction that the claim is false, and hence $z_1\neq u(\sigma z_2)^k$. Therefore, there are $\tau\neq \tau'$ in $\N$ and a word $y$, such that $\tau'y$ is a suffix of $\sigma z_2$, and 
	$$z_1=x\tau y(\sigma z_2)^r, \ (\sigma z_2)^k= x'\tau' y(\sigma z_2)^r.$$
	Clearly, $\tau<\tau'$ since otherwise, $\tau'<\tau$, and we get that 
	$w=0^lz_1\sigma z_2=0^lx\tau y (\sigma z_2)^{r+1}$. However, if we assume that $\tau'<\tau$, $w'=(\sigma z_2)^r0^lx\tau y$ is lexicographically larger than $w$, in contradiction to $w\in L_n$.
\end{proof}

\begin{corrolary}
	\label{z1-suffix}
	Assume that $w=0^lv\in L_n$ and $|w|=n$. Write $w=0^lz_1\sigma z_2$ where $\sigma$ is the first digit in such that $0^{l+|z_1|}(\sigma+1)z_2$ is lexicographically maximal among its rotations. Then, there are words $x,y$ such that $z_2=xy$, $w=0^ly(\sigma xy)^{r+1}$ and $z_1=y(\sigma xy)^r$.
\end{corrolary}
\begin{proof}
	This is a consequence of the previous Lemma and the fact that $|0^lz_1|=|x(\sigma z_2)|^m$.
\end{proof}

\begin{lemma}
	\label{concatenation}
	If $uv=vu$ and $u,v\neq\varepsilon$, then there is some word $w$, such that $u,v\in\{w\}^*$.
\end{lemma}

\begin{proof}
	By induction on $|u|+|v|$. If $|u|=|v|$, $u=v$ and we are done. Otherwise, assume w.l.o.g. that $|u|>|v|$ and write $u=vx$ (since $uv=vu$). Then, $ux=vxv=vvx=vu$. We see that $xv=vx$. By the induction hypothesis, $x=w^k$ and $v=w^l$. Hence, $u=w^{l+k}$ as required.
\end{proof}

\begin{lemma}
	\label{not-in-Ln}
	Let $w=0^ly(x0^ly)^{r+1}$ be an $n$-length word such that $y\notin \{0\}^*$. Then, $w\notin L_n$.
\end{lemma}

\begin{proof}
	Assume for a contradiction that $w$ is a key-word of length $n$, and take a maximal $t\in \N$ such that $x0^ly=x'(0^ly)^{t+1}$. First, we note that $x'\neq \varepsilon$. Indeed, if $x'=\varepsilon$, then $w=(0^ly)(0^ly)^{(t+1)(r+1)}$, a periodic word, and then $w\notin L_n$.
	
	Now we claim that $|x'|<|0^ly|$. For verifying this claim, assume that $|x'|\geq |0^ly|$ and write $x'=x'_1x'_2$, where $|x'_2|=|0^ly|$. By maximality of $t$, $x'_2\neq 0^ly$, and since $w\in L_n$, $x'_2<_{lex} 0^ly$. Therefore, $$w'=(x0^ly)^rx_1'x_2'(0^ly)^{t+1}0^ly$$ is a rotation of $w$ which is lexicographically larger then $w$, in contradiction to $w\in L_n$.
	
	To summary our conclusions, we have $w=0^ly(x'(0^ly)^{t+1})^{r+1}\in L_n$, and $|x'|<|0^ly|$. Write $0^ly=z_1z_2$ where $|x'|=|z_2|$. Therefore, $$w=z_1z_2(z_2(z_1z_2)^{t+1})\dots(z_2(z_1z_2)^{t+1}).$$ 
	
	We look now at a rotation of $w$, $w'=(z_2(z_1z_2)^{t+1})\dots(z_2(z_1z_2)^{t+2})$. Since $w\in L_n$, $w$ is lexicographically larger than $w'$ and in particular,
	$(z_1z_2z_2(z_1z_2)^{t+1})\geq_{lex}(z_2(z_1z_2)^{t+2})$ which implies that
	$z_1z_2z_2\geq_{lex}z_2z_1z_2$, and hence
	$$z_1z_2\geq_{lex}z_2z_1.$$
	In addition, $z_2z_1z_2$ is a suffix of $w$ while $z_1z_2z_2$ is a subword of $w$. Hence, as $w\in L_n$ we have, $z_2z_1z_2\geq_{lex} z_1z_2z_2$, and hence
	$$z_2z_1\geq_{lex}z_1z_2.$$
	As a result, $z_2z_1=z_1z_2$m and then by Lemma \ref{concatenation}, $z_1=z^{l_1}$ and $z_2=z^{l_2}$ for some non empty word $z$. Therefore, $w=z^m$ for some $z>0$ in contradiction to $w\in L_n$. 
	
	
\end{proof}

\begin{proposition}
	Assume that $v0^l\in\mathbb{N}^n$ corresponds to $w=0^lv\in L_n$ where $|w|=n$ and $l>0$. Then, $v0^l$ is a subword of $w^\infty$. 
\end{proposition}

\begin{proof}
	Write $w=0^lz_1\sigma z_2$ where $\sigma\in \N$ is the first digit in $w$ so that  $0^{l+|z_1|}(\sigma+1)z_2$ is lexicographically maximal among its rotations. Note that such a digit exists since the last digit in $w$ satisfies this requirement. Hence, $v=z_1\sigma z_2$.
	
	By Corollary \ref{z1-suffix}, $z_2=xy$ and $z_1=y(\sigma xy)^r$. Now, since $|0^{l+|z_1|}(\sigma+1)z_2|=n$ and $0^{l+|z_1|}(\sigma+1)z_2$ is lexicographically maximal among its rotations, $0^{l+|z_1|}(\sigma+1)z_2=(w')^{k+1}$ where $w'\in L_n$. Note that $0^{l+|z_1|}$ is a prefix of $w'$. We consider three possibilities
	
	\begin{description}
		
		\item{Case 1.}  $\sigma z_2\in L_n$. We show that in this case, $v0^l$ is a subword of $b^*(b^*(w'))(b^*(w'))w'$, which is a subword of $w^\infty$.
		
		$b_1(w')=w'^{k+1}=0^{l+|z_1|}(\sigma+1)z_2$. Hence, $b(w')=b_3(b_2(0^{l+|z_1|}(\sigma+1)z_1))=\sigma z_2$. Since $\sigma z_2\in L_n$, $b(w')=b^*(w')=\sigma z_2$ and in particular, $|(\sigma z_2)^{m+1}|=n$ for some $m\in \N$. Observe that $|z_1|\leq |\sigma z_2|^m$ and use Lemma \ref{z1} to conclude that $z_1$ is a suffix of $(\sigma z_2)^m$.  
		
		By invoking Proposition \ref{1-back} several times, $b^*(\sigma z_2)= u(\sigma z_2)^m$ for some $u$. Hence, $v0^l=z_1\sigma z_2 0^l$ is a subword of 
		$$b^*(b^*(w'))b^*(w')b(w')=u(\sigma z_2)^{m+1}0^{l+|z_1|}x'.$$
		
		
	\end{description}
	
	Before we deal with the other cases, we note that $b_1(\sigma z_2)=b_1(\sigma xy)=x'y(\sigma xy)^{r+1}$ for some $x'$ that satisfies $|x'|=l>0$.
	
	\begin{description}
		\item{Case 2.} $\sigma z_2\notin L_n$  and $x'\neq 0^l$. We show that in this case, $v0^l$ is a subword of $b^*(w')w'$, which is a subword of $w^\infty$.
		
		Recall that $b(w')=\sigma z_2$ which is, by assumption, not a key-word. Since $x'\neq 0^l$, several invocations of Proposition \ref{1-back} imply that $b^*(\sigma z_2)=x''y(\sigma z_2)^{r+1}$. Since $v=z_1\sigma z_2=y(\sigma z_2)^{r+1}$, we get that $v0^l$ is a subword of 
		$$b^*(w')w'=x''y(\sigma z_2)^{r+1}0^{l+|z_1|}u.$$
		
		
		\item{Case 3.} $\sigma z_2\notin L_n$  and $x'=0^l$. In this case, $b_1(\sigma z_2)$ = $0^ly(\sigma xy)^{r+1}$. Note that $w=0^ly(x''0^ly)^{r+1}$ and use Lemma \ref{not-in-Ln} to obtain a contradiction.
		
	\end{description}
	
	
\end{proof}

\begin{theorem}
	$w^\infty$ is an infinite de Bruijn sequence.
\end{theorem}

\begin{proof}
	According to Propositions \ref{}, every $n$-sequence is a subword of $w^\infty$. By the ``onion theorem" and by the pigeonhole principle, every $n$-sequence appears only once at $w^\infty$.
\end{proof}


\section{Constructing an Infinite de Bruijn Cycle}

A key word is an $n$-length word that is (arabic) maximal among its rotations. Let $kw_0,kw_1,kw_2,\dots$ be an enumeration of all key-words, ordered lexicographically. Let $C_m$ be the cycle of $kw_m$. We order the elements of $C_m$ as follows:  if $kw_m=0^l(\sigma+1)w$, then $w0^l(\sigma+1)$ is the first sequence in $C_m$, and each word $w'$ is followed by $R(w')$. The last word in $C_m$ is $(\sigma+1)w0^l$.\\

For each $m<n$ we define a de Bruijn sequence $D_m$, over the words $\bigcup_{i=0}^m C_i$ as follows: 
\begin{itemize}
	\item $D_0=0^n$.
	\item If $kw_{m+1}=0^l(\sigma+1)w$, then $D_{m+1}$ is obtained by inserting the sequence $C_{m+1}$ after the word $\sigma w 0^l\in C_m$.
\end{itemize}

%\begin{lemma}
%Every $D_m$ is a de Bruijn sequence over $\bigcup_{i=0}^m C_i$, and   $\forall m$ $\exists k$ so that if $k\leq r$, then $D_m$ is a prefix of $D_r$. 
%\end{lemma}

\begin{definition}
	For $w,w\in \N^n$ and $m\in \N$, write $w<_m w'$ if $w$ appears before $w'$ in $D_m$. Write $<=\bigcup_{i=0}^\infty$.
\end{definition}

For a word $w$, $max(w)$ is the maximal number in $w$.

\begin{lemma}
	If $w<w'$, then $max(w)\leq max(w')$.
\end{lemma}

\begin{corrolary}
	$<$ defines an infinite de Bruijn sequence.
\end{corrolary}

\begin{proof}
	By the previous Lemma, each word is preceded by finitely many words thus $<$ defines an infinite sequence. Since each $D_m$ is a de Bruijn sequence and since $<_m\subseteq<_{m+1}$, the sequence defined by $<$ is a de Bruijn sequence. 
\end{proof}

Let $D$ be the infinite de Bruijn sequence defined by $<$. 

\begin{theorem}
	For a  word $\sigma w$ in $D$, let $nxt(\sigma w)$ be the successor of $\sigma w$ in $D$. Then,
	$$nxt(\sigma w)=\left\{ \begin{array}{ll}
	w(\sigma+1) & \text{if }last((\sigma+1)w);\\
	w0          & \text{if } last(\sigma w);\\
	w\sigma     & \text{otherwise}
	\end{array}\right.$$
\end{theorem}

\begin{definition}
	If $kw_m=0^l(\sigma+1)w$, we write $first(w0^l(\sigma+1))$, $key(0^l(\sigma+1)w)$ and $last((\sigma+1)w0^l)$. In addition, for a cycle $C_m$, $first(C_m)=w\in C_m$ so that $first(w)$,  $key(C_m)=kw_m$ and $last(C_m)=w\in C_m$ so that $last(w)$. 
\end{definition}

\begin{lemma}
	For any $w'\in C_m$, $first(C_m)\leq w'\leq last(C_m)$.
\end{lemma}
\begin{proof}
	This the way we ordered the cycles.
\end{proof}



\begin{definition}
	We say that $C_{i_0}C_{i_1}\dots C_{i_k}$ is a sequence of cycles, if for every $j<k$, $nxt(last(C_{i_j}))=first(C_{i_{j+1}})$.
\end{definition}

\begin{lemma}
	\label{sequence}
	Let $C_{i_0}C_{i_1}\dots C_{i_k}$ be a sequence of cycles. Write, $key(C_{i_0})=0^l(\sigma)w$ where $\sigma\neq 0$. Then, for every $j\leq k$, $key(C_{i_j})=0^l(\sigma+j)w$.
\end{lemma}

\begin{proof}
	Assume by induction that $key(C_{i_j})=0^l(\sigma+j)w$ for $j<k$. Hence, $last(C_{i_j})=(\sigma+j)w0^l$. Thus, $nxt((\sigma+j)w0^l)=w0^l(\sigma+j+1)$ or $nxt((\sigma+j)w0^l)=w0^{l+1}$. Since $\neg(first(w0^{l+1}))$, we have 
	$$nxt((\sigma+j)w0^l)=w0^l(\sigma+j+1)=first(C_{i_{j+1}}).$$
	We conclude that $key(C_{i_{j+1}})=0^l(\sigma+j+1)w$.
\end{proof}

\begin{lemma}[The parentheses property]
	\label{parenthesis}
	For any two cycles $C_k$ and $C_m$, one of the following occur
	\begin{itemize}
		\item $last(C_k)<first(C_m)$ or $last(C_m)<first(C_k)$.
		
		\item $first(C_k)<first(C_m)\leq last(C_m)<last(C_k)$ or \\
		$first(C_m)<first(C_k)\leq last(C_k)<last(C_m)$.
	\end{itemize}
\end{lemma}
\begin{proof}
	This concluded by the way $D_{m+1}$ is obtained from $D_m$.
\end{proof}

\begin{definition}
	We say that $C_m$ is embedded in $C_k$, if $C_m=C_k$ or $first(C_k)<first(C_m)\leq last(C_m)<last(C_k)$.\\ 
	In addition, $C_m$ is said to be immediately embedded in $C_k$ if there is no $C_l$ such that $C_m$ is embedded in $C_l$ and $C_l$ is embedded in $C_k$.\\ 
	We define by inductively the statement: ``$C_m$ is $r$-embedded in $C_k$":
	\begin{itemize}
		\item $C_m$ is $0$-embedded in $C_k$ if $C_m=C_k$.
		\item $C_m$ is $r$-embedded in $C_k$ if there is a cycle $C_l$ such that $C_m$ is $r-1$-embedded in $C_l$ and $C_l$ is immediately embedded in $C_k$. 
	\end{itemize} 
\end{definition}

\begin{lemma}
	\label{embedding-properties}
	Assume that $C_m$ is immediately embedded $C_k$. Write $key(C_m)=0^i(\sigma+1)0^jw$ where $w$ does not starts with $0$. Then,
	\begin{itemize}
		\item $key(C_k) =0^{i+1+j}w$.
		\item If $u\in C_k$ and $last(C_m)<u$, then $u=0^{j_2}w0^{i+1+j_1}$ where $j_1+j_2=j$. 
	\end{itemize}
\end{lemma}


\begin{proof}
	To prove the first item, note that since $0^i(\sigma+1)0^jw$ is maximal among its rotations, the same holds for $0^{i+1+j}w$. Thus, we need to show that $0^{i+1+j}w\in C_k$. 
	
	Take a maximal sequence of cycles that begins in $C_m$ and let $C_{m+r}$ be the last cycle in this sequence. By lemma \ref{sequence}, $key(C_{m+r})=0^i(\sigma+1+r)0^jw$. By the parentheses property, $C_{m+r}$ is immediately embedded in $C_k$. And finally, by the maximality of the sequence of cycles, $nxt(last(C_{m+r}))\in C_k$.
	
	As a result, we have:
	$$nxt(last(C_{m+r}))=nxt((\sigma+1+r)0^jw0^i)\in\{\ 0^jw0^i0, 0^jw0^i(\sigma+2+r)\}.$$ If $nxt((\sigma+1+r)0^jw0^i)= 0^jw0^i(\sigma+2+r)$, then $last((\sigma+2+r)0^jw0^i)$ which implies $first(0^jw0^i(\sigma+2+r))$. But $first(nxt(last(C_{m+r})))$ contradicts the maximality of our sequence of cycles. Hence, $nxt((\sigma+1+r)0^jw0^i)= 0^jw0^i0\in C_k$. Now, $0^{i+1+j}w$ is a rotation of $0^jw0^i0$ thus $0^{i+1+j}w\in C_k$ as required.
	
	For proving the second item, we note that $last(C_k)=w0^{i+1+j}$. As we have seen, the first element in $C_k$ that follows $C_m$ (and follows $C_{m+r}$) is $nxt(last(C_{m+r}))= 0^jw0^i0$. As a result, if $u\in C_k$ and $last(C_m)<u$, then 
	$$0^jw0^i0\leq u \leq w0^{i+1+j}$$
	which implies that $u=0^{j_2}w0^{i+1+j_1}$
\end{proof}

\begin{corrolary}
	\label{embedding-deletes-zeroes}
	Assume that $C_m$ is $r$-embedded in $C_k$. Write $key(C_m)=uv$ where $u$ is the minimal prefix of $key(C_m)$ that includes $r$ non-zero numbers. Then, $key(C_k)=0^{|u|}v$.
\end{corrolary}

\begin{proof}
	By $r$ invocations of the first item of the previous lemma.
\end{proof}



\begin{lemma}
	\label{both-embedded}
	Assume that $last(C_k)<first(C_m)$ and $C_k,C_m$ are both immediately embedded in a cycle $C$. Write $key(C_k)=0^lw$ where $w$ starts with a non-zero letter. Then, $key(C_m)=0^lw'$ where $w<_{lex}w'$.
\end{lemma}
\begin{proof}
	First, if there is a sequence of cycles from $C_k$ and $C_m$, the claim follows from Lemma \ref{sequence}. Otherwise, there is some $u\in C$ such that
	$$last(C_k)<u<first(C_m).$$
	Consider a  maximal sequence of cycles that starts with $C_k$. This sequence ends in some cycle $C_{k'}$. Similarly, consider a maximal sequence of cycles that ends in $C_m$ and let $C_{m'}$ be the first element in this sequence. Therefore,
	$$last(C_{k'}) <v<first(C_{m'})\Longrightarrow v\in C.$$
	
	Write $key(C_k)=0^l(\sigma+1)w_1$ (namely, $w=(\sigma+1)w_1$) and write  $w_1=0^jw_2$ where $w_2$ starts with non-zero letter. Hence, 
	\begin{equation}
	\label{ck}
	key(C_k)=0^l(\sigma+1)0^jw_2.
	\end{equation}
	By Lemma \ref{sequence}, $key(C_{k'})=0^l(\sigma+1+r)0^jw_2$, and hence $last(C_{k'})=(\sigma+1+r)0^jw_20^l$. Let $u_1$ be the successor of $last(C_{k'})$. Thus,
	$$u_1=0^jw_20^{l+1}.$$
	Now, let $u_2$ be the predecessor of $first(C_{m'})$. Thus,
	$$u_2=0^{j_2}w_20^{l+1+j_1}, \text{ where } j_1+j_2=j.$$
	In addition, since $C_{m'}$ is embedded in $C$, $u_2\neq last(C)$ thus $j_2>0$. As a result, $$first(C_{m'})=0^{j_2-1}w_20^{l+1+j_1}1.$$
	We get that $key(C_{m'})=0^{l+1+j_1}10^{j_2-1}w_2$. Hence, by Lemma \ref{sequence}, 
	\begin{equation}
	\label{cm}
	key(C_m)=0^{l+1+j_1}(1+t)0^{j_2-1}w_2.
	\end{equation}
	By Equation \ref{ck}, $w=(\sigma+1)0^{j}w_2$. By Equation \ref{cm}, $w'=0^{1+j_1}(1+t)0^{j_2-1}w_2$. We see that indeed $w<_{lex}w'$.
\end{proof}

\begin{lemma}
	\label{first-ordering}
	If $k<m$, then $first(C_k)<first (C_m)$.
\end{lemma}

\begin{proof}
	Since $<$ is a linear ordering, we can prove an equivalent statement: $$first(C_k)<first(C_m) \Longrightarrow k<m.$$ We take such cycles $C_k$ and $C_m$. By the parentheses property, either $C_m$ is embedded in $C_k$, or $C_m$ is entirely after $C_k$. If $C_m$ is embedded in $C_k$, by Corollary \ref{embedding-deletes-zeroes} we get that $k<m$. It is left to deal with the case that $last(C_k)<first(C_m)$. We consider two cases.
	\begin{description}
		\item[Case 1.] $C_k$ and $C_m$ are both embedded in some cycle $C$.
		
		In this case, we can find cycles $C_{k'}$ and $C_{m'}$ such that
		\begin{enumerate}
			\item $C_k$ is embedded in $C_{k'}$ and $C_{m}$ is embedded in $C_{m'}$
			
			\item $C_{k'}$ and $C_{m'}$ are immediately embedded in $C$.
			
		\end{enumerate}
		
		By item 1 and Lemma~\ref{embedding-deletes-zeroes}, we can write $key(C_k)=uv$ and $key(C_{k'})=0^{|u|}v$. Write 
		$$v=0^lv_1$$ 
		where $v_1$ starts with a non-zero letter thus $key(C_{k'})=0^{|u|+l}v_1$. By item~2 and Lemma~\ref{both-embedded}, $key(C_{m'})=0^{|u|+l}v_2$ where 
		$$v_1<_{lex}v_2.$$
		
		Write $v_2=0^r v_2'$ where $v_2'$ starts with a non zero letter. Hence, $$v_1=xv_1'$$ 
		where $|x|=r$, $|v_1'|=|v_2'|$, and 
		$$v_1'<_{lex} v_2'.$$
		Since $key(C_{m'})=0^{|u|+l}v_2 =0^{|u|+l+r}v_2'$, by item~2 and Lemma \ref{embedding-deletes-zeroes}, $key(C_m)=u'v_2'$. Recall that $key(C_k)=uv=u0^lv_1=u0^lxv_1'$. Since $v_1'<_{lex}v_2'$, $key(C_k)<_{lex}key(C_m)$ thus $k<m$ as required.
		
		\item[case 2.] There is no cycle $C$ such that $C_k$ and $C_m$ are embedded in $C$.
		
		Take a cycle $C_{k'}$ such that $C_k$ is $r_1$-embedded in $C_{k'}$ and $r_1$ is maximal with respect to this property. Similarly, take a cycle $C_{m'}$ such that $C_{m}$ is $r_2$-embedded in $C_{m'}$ and $r_2$ is maximal as possible. By the parentheses property, $C_{k'}$ is entirely before $C_{m'}$. $C_{k'}$ and $C_{m'}$ are not embedded in any cycle so there is a sequence of cycles from $C_{k'}$ to $C_{m'}$. Write $key(C_k)=u(\sigma+1)v$ such that $key(C_{k'})=0^{|u|}(\sigma+1)v$ (by Lemma \ref{embedding-deletes-zeroes}). Hence, $key(C_{m'})=0^{|u|}(\sigma+1+r)v$ where $r>1$ (by Lemma \ref{sequence}). Finally, by Lemma \ref{embedding-deletes-zeroes} we get $key(C_m)=u'(\sigma+1+r)v$. We see that $key(C_k)=u(\sigma+1)v<_{lex}u'(\sigma+1+r)v=key(C_m)$. Therefore, $k<m$.
		
		
	\end{description}
\end{proof}








\begin{theorem}
	For any word $\tau w$, $\tau w<(\tau+1)w$.
\end{theorem}

\begin{proof}
	Fix an arbitrary cycle $C_m$. We show that if $(\tau+1)w\in C_m$. Then, $\tau w<(\tau+1)w$. We start by showing this fact for $last(C_m)$. Write $key(C_m)=0^l(\sigma+1)w'$ and hence, $last(C_m)=(\sigma+1)w'0^l$ and $first(C_m)=w'0^l(\sigma+1)$. Write $pre=nxt^{-1}$ and note that $pre(w'0^l(\sigma+1))=\sigma w'0^l$. Hence, we get that 
	$$\sigma w' 0^l<(\sigma+1)w'0^l.$$
	
	Now we deal with the general case in which $(\tau+1)w\neq last(C_m)$. We write 
	$key(C_m)=0^l(\sigma+1)w_1(\tau+1)w_2$, where $$(\tau+1)w=(\tau+1)w_20^l(\sigma+1)w_1.$$ Let $C_k$ be the cycle of $\tau w$. Since every rotation of $\tau w$ is lexicographically smaller than some rotation of $(\tau+1)w$, we have $key(C_k)<_{lex}key(C_m)$. Hence, $k<m$ and by Lemma \ref{first-ordering}, we get 
	$$first(C_k)<first(C_m).$$
	
	Now, if $last(C_k)<first(C_m)$ then every element of $C_k$ precedes every element of $C_m$ and we are done. Otherwise, by the parentheses property, $C_m$ is embedded in $C_k$. Note that $|(\tau+1)w|_0-|\tau w|_0\in\{0,1\}$. Use corollary \ref{embedding-deletes-zeroes} to conclude that $|(\tau+1)w|_0-|\tau w|_0=1$ and that $C_m$ is immediately embedded in $C_k$. Moreover, note that as $|(\tau+1)w|_0-|\tau w|_0=1$, we have $\tau=0$.
	
	According to our conclusions, we can write: $key(C_m)=0^l(\sigma+1)w_11w_2$. We write $w_1=0^jw_1'0^i$ and $w_2=0^rw_2'$ where $w_1'$ and $w_2'$ do not start or end with zero. We have,
	$$key(C_m)=0^l(\sigma+1)0^jw_1'0^i10^rw_2'.$$
	Assume for a contradiction that $(\tau+1)w<\tau w$. Recall that $\tau w\in C_k$ and that $C_m$ is immediately embedded in $C_k$, and conclude that $last(C_m)<\tau w\leq last(C_k)$. Therefore, by Lemma \ref{embedding-properties}, $\tau w=0^{j_2}w_1'0^i10^rw_2'0^{l+1+j_1}.$ Thus, we have:
	\begin{equation}
	\label{1}
	0^{j_2}w_1'0^i10^rw_2'0^{l+1+j_1}=0^{r+1}w_2'0^l(\sigma+1)0^jw_1'0^i.
	\end{equation}
	Since also $|0^{j_2}w_1'0^i10^rw_2'0^{l+1+j_1+1}|_1 = |0^{r+1}w_2'0^l(\sigma+1)0^jw_1'0^i|_1$ we get $\sigma+1=1$. Hence,
	
	\begin{equation}
	\label{2}
	key(C_m)=0^l10^jw_1'0^i10^rw_2'
	\end{equation}
	
	and Equation \ref{1} can be rewritten as follows:
	
	\begin{equation}
	\label{3}
	0^{j_2}w_1'0^i10^rw_2'0^{l+1+j_1}=0^{r+1}w_2'0^l10^jw_1'0^i.
	\end{equation}
	
	For the rest of the proof we assume that $w_1'\neq\varepsilon$ and $w_2'\neq \varepsilon$. The other cases are dealt similarly\footnote{really! I checked !}.
	By deleting the initial and final segments of zeroes, we get from Equation \ref{3},
	\begin{equation}
	\label{4}
	j_2=r+1, \ \ w_1'0^i10^rw_2'=w_2'0^l10^jw_1'.
	\end{equation}
	Now, by Equation \ref{2}, 
	\begin{equation}
	\label{5}
	0^l10^jw_1'0^i10^rw_2'\geq_{lex} 0^i10^rw_2'0^l10^{j}w_1'.
	\end{equation}
	By combining equations \ref{4} and \ref{5}, we get 
	\begin{equation}
	0^l10^j\geq_{lex}0^i10^r.
	\end{equation}
	Hence, $j\leq r$ and in particular, $j_2\leq r$ in contradiction to Equation \ref{4}.
\end{proof}


\end{document}

