\documentclass{article} %\documentclass[letterpaper, 10 pt,
% conference]{ieeeconf}

%\IEEEoverridecommandlockouts        % This command is only needed if you want
%to use the \thanks command
%\overrideIEEEmargins                        % See the \addtolength command
%later in the file to balance the column lengths


% Name clashes workaround
%\makeatletter
%\let\IEEEproof\proof
%\let\IEEEendproof\endproo f
%\let\proof\@undefined
%\let\endproof\@undefined
%\makeatother




\usepackage{epsfig} \usepackage{amsmath} \usepackage{amssymb}
\usepackage{amsthm} \usepackage{hyperref} \usepackage{tikz}
\usetikzlibrary{automata,arrows,decorations.pathmorphing,decorations.markings,positioning} \usetikzlibrary{shapes} %\usepackage{balance}
\usepackage{scrextend} \addtokomafont{labelinglabel}{\sffamily}
\usepackage{refcount}


% Theorem like env.
\newtheorem{theorem}{Theorem} \newtheorem{proposition}[theorem]{Proposition}
\newtheorem{claim}[theorem]{Claim} \newtheorem{lemma}[theorem]{Lemma}

\newtheorem{reptheorem}{Theorem} \newenvironment{repeatedtheorem}[1] {
	\setcounterref{reptheorem}{#1} \addtocounter{reptheorem}{-1} \begin{reptheorem}
	}{ \end{reptheorem} }


\theoremstyle{definition} \newtheorem{definition}[theorem]{Definition}
\newtheorem{conjecture}[theorem]{Conjecture}
\newtheorem{observation}[theorem]{Observation}
\newtheorem{example}[theorem]{Example} \newtheorem{remark}[theorem]{Remark}
\newtheorem{corrolary}[theorem]{Corrolary}



%\newcommand{\RotateLeft}{rlft}
\DeclareMathOperator{\RotateLeft}{RotLft}



%------------------------------------------- preamble
%---------------------------

\newcommand{\R}{{\mathbb{R}}} \newcommand{\N}{{\mathbb{N}}}
\newcommand{\Z}{{\mathbb{Z}}}

\newcommand{\REF}[2]{#1~\ref{#2}}

\newcommand{\T}[1]{\langle{#1}\rangle} \newcommand{\rr}[2]{R^{#2}({#1})}
\newcommand{\rl}[2]{R^{-{#2}}({#1})}


\newcommand{\AorB}[2]{\bigl(#2\text{?}A(#1)\text{:}B(#1)\bigr)}

\usepackage{amssymb} \usepackage{algpseudocode,algorithm} \usepackage{subfigure}

\newcommand{\nega}   [1]{\overline{#1}} \newcommand{\companion}{comp}
\newcommand{\shlsame}[1]{shl\text{-}same(#1)} \newcommand{\shlopp}
[1]{shl\text{-}opp(#1)} \newcommand{\shlone} [1]{shl\text{-}one(#1)}
\newcommand{\shlzero}[1]{shl\text{-}zero(#1)} \newcommand{\idx}    [1]{idx(#1)}


%\pagestyle{plain}


\title{Efficient Constructions of Prefer-Max Sequences and of Infinite de Bruijn
	Sequences}

\author{}

%%%%%%%%%%%%%%%%%%%%%%%%%%%%%%%%%%%%%%%%%%%%%%%%%%%%%%%%%%%%%%%%%%%%%%%%%%%%%%%%%%%%%%%%%%%%%%%%%%%%

\begin{document} \maketitle
	
	\section{Introduction}
	
	A $k$-ary de Bruijn sequence of span $n$ is a cyclic sequence of length $k^n$
	in which each $k$-ary string of length $n$ appears exactly once as a substring.
	A well known construction of such a sequence, first proposed by
	Ford~\cite{Ford1957}, can be described as follows:
	
	\begin{algorithm} Repeatedly append the maximal symbol over the alphabet
		$0,\dots,k-1$ such that the word formed by the $n$ most recent symbols is new.
		In the first $n$ steps, act as if there are $n$ invisible zeros before the
		sequence. Stop after appending $n$ consecutive zeros. \caption{The
			$(k,n)$-prefer-max sequence.} \label{pref-max} \end{algorithm}
	
	
	For example, if we choose $k=3$ and $n=3$ we get the sequence:
	$$\langle2,2,2,1,2,2,0,2,1,1,2,1,0,2,0,1,2,0,0,1,1,1,0,1,0,0,0\rangle.$$
	
	
	An obvious limitation of the above construction is, of course, that, when $n$
	and $k$ grow, the amount of time and memory needed to produce the next symbol
	grows exponentially. Therefore, this construction is not practical for
	generating de Bruijn sequences of large alphabets and spans. It is used here as
	a mathematical definition of the sequence. More formally, a shift rule for a de
	Bruijn sequence of order $n$ is a function that maps each word of length $n$ to
	the symbol that follows that word as a substring in sequence. The above is an
	inefficient shift rule for the prefer-max sequence. The first (to the best of
	our knowledge) efficiently computable shift rule for the prefer-max sequence is
	one of the contributions of this paper.
	
	Note that the $(n,k)$-prefer-max sequence begins with a prefix (the first $19$
	symbols) that contains all the words in $\{0,1,2\}^n \setminus \{0,1\}^n$  (the
	words that contain the symbol $2$) as substrings. Note also that the rest of
	the sequence, after this prefix, is exactly the sequence that we would have
	obtained if we ran the construction with the alphabet $k=2$ (and the same $n$).
	The first result of this paper, that we call the Onion theorem, is that this is
	not a coincidence:
	
	\begin{theorem}[Onion] The $(k-1,n)$-prefer-max sequence is a suffix of the
		$(k,n)$-prefer-max sequence. \label{thm:onion} \end{theorem}
	
	The observation made in the last theorem directed us to examine the reverse of
	the prefer-max sequence. Specifically, the structure identified in
	Theorem~\ref{thm:onion} means that, if we manage to construct the sequence
	backwards, we do not have to specify $k$ ahead of time. Instead, we can add
	symbols forever while $k$ grows in layers: we first write $0^n$, then a word
	whose substrings of length $n$ are $\{0,1\}^n$, then extend with other words in
	$\{0,1,2\}^n$, and so on. The second contribution of this work is then an
	efficient construction of this reversed sequence. This is formulated in
	Algorithm~\ref{inf-construction}. We say that this construction is efficient
	because each step can be computed in $O(n^2)$ operations by enumerating all the
	$n$ rotations of $\sigma w$ and comparing them lexicographically (reading each
	word from right to left).
	
	
	\begin{algorithm}[!h] Append $n$ zeros. Then, repeatedly, let $\sigma w$ be the
		word formed by the $n$ most recent symbols. Append $\sigma+1$ if $(\sigma+1)w$
		is of the form $u0^l$ where $0^lu$ is maximal in right-to-left lexicographic
		order among all of its rotations. If $(\sigma+1)w$ is not of that form and
		$\sigma w$ is, append the symbol $0$. Otherwise, if neither $(\sigma+1)w$ nor
		$(\sigma)w$ are of that form, append $\sigma$. \caption{An infinite de Bruijn
			sequence.} \label{inf-construction} \end{algorithm}
	
	The fact that this generates the reverse of the prefer-max sequence is
	formulated in the following theorem:
	
	\begin{theorem} The first $k^n$ symbols generated by
		Algorithm~\ref{inf-construction} form the reverse of the sequence generated by
		the prefer-max construction with the parameters $k$ and $n$. \end{theorem}
	
	In particular, the above theorem says that the sequence generated by
	Algorithm~\ref{inf-construction} is a de Bruijn sequence, i.e., that each word
	in $\N^n$ appears exactly once as a substring.
	
	While our main focus is on analyzing the reversed sequence, we also have a
	result about the prefer-max sequence itself. Our result gives the first (to the
	best of our knowledge) efficiently computable shift rule for that sequence:
	
	\begin{algorithm} Append $n-1$ zeros and then the symbol $k$. Then, repeatedly,
		let $\sigma w$ be the word formed by the $n$ most recent symbols. Append the
		symbol $\sigma-1$ if $\sigma w$  is of the form specified in
		Algorithm~\ref{inf-construction}. Otherwise, if there is a  $\sigma'$ such that
		$\sigma'w$ is of that form, append the maximal such $\sigma'$. Otherwise, if
		there is no such $\sigma'$, append the symbol $\sigma$.
		
		\caption{An efficient construction of the prefer-max sequence.}
		\label{efficient-pref-max} \end{algorithm}
	
	Each step of Algorithm~\ref{efficient-pref-max} can be computed in $O(n^2)$
	time because it suffices to examine, as candidates for $\sigma'$, only symbols
	that are in $\sigma w$ or are smaller by one than a symbol in $\sigma w$, as we
	will establish later in the paper. The fact that this is indeed the prefer-max
	sequence is stated in the following theorem:
	
	\begin{theorem} The sequence generated by Algorithm~\ref{efficient-pref-max} is
		the prefer-max sequence with parameters $k$ and $n$. \end{theorem}
	
	
	A central tool to our analysis of the sequences described above is an
	alternative formulation of the prefer-max sequence. For this, we need to
	introduce some language, as follows.
	
	The $n$ dimensional infinite de Bruijn graph is the graph $G_{n}=\T{V,E}$ where
	$V=\N^n$ and $E=\{ \T{\sigma_1 w, w \sigma_2} \colon \sigma_1,\sigma_2\in\N, w
	\in \N^{n-1}\}$. The de Bruijn sequences, defined in the beginning of this
	paper, are in one-to-one correspondence with the Hamiltonian cycles in the de
	Bruijn graph, as follows: (1) Concatenating the first letter of each vertex in
	a cycle produces a de Bruijn sequence of order $n$; and (2) A cycle can be
	constructed from a de Bruijn sequence $\T{\sigma_1,\dots,\sigma_{2^n}}$ of
	order $n$ by visiting the vertex $\sigma_1\cdots\sigma_n$, then
	$\sigma_2\cdots\sigma_{n+1}$ and so on.
	
	Let $rtl(\sigma_1\sigma_2\dots\sigma_n) = \sigma_2\dots\sigma_n\sigma_1$ denote
	the rotation of a word to the left and let $rtl^i(w)$ denote a rotation of $i$
	letters to the left. A \emph{necklace} in the de Bruijn graph $G_n$ is the set
	$\{rtl^i(w)\}_{i=1}^n$ where $w \in \N^n$, i.e., it is an equivalence class of
	$n$-character strings over $\N$ when strings that are rotations of each other
	are considered equivalent. A $w\in\N^n$ is called a \emph{key word} if it is
	larger in right-to-left lexicographic order than all of its rotations (all the
	words in its necklace).  Note the resemblance of this to the notion of Lyndon
	words~\cite{Lyndon1954}, on which we will elaborate later.
	
	We are now ready to introduce another construction:
	
	
	\begin{algorithm}[!h] Start with the one element sequence $\T{0^n}$. The
		sequence at the $i$th step is constructed by adding words to the previous
		sequence as follows: Let $k_i$ be the $i$th key word in right-to-left
		lexicographic order. Let $l$ be the number of leading zeros in $k_i$ ($l$ may
		be zero) and let $\sigma \in \N$ and $w\in\N^{n-l-1}$ be such that
		$k_i=0^l(\sigma+1)w$. Now, add the sub-sequence
		$\T{rtl^i(w0^l(\sigma+1))}_{i=0}^{n-1}$ between the word $\sigma w 0^l$ and its
		follower in the previously constructed sequence. \caption{A necklace joining
			construction.} \label{necklace-joining} \end{algorithm}
	
	To following proposition establishes that the above construction describes an
	infinite sequence of words:
	
	\begin{proposition} For every $l>0$ there is $m>0$ such that for every $i>m$
		the insertions of words at the $i$th step of Algorithm~\ref{necklace-joining}
		are all after the $l$th element of the sequence. \end{proposition}
	
	Let $S=\T{w_i}_{i=0}^\infty$ be the infinite sequence all whose prefixes agree
	with infinitely many prefixes of the finite sequences generated at the steps of
	Algorithm~\ref{necklace-joining}. Since every word is a rotation of some key
	word, we have that this sequence contains all the words in $\N^n$. Since there
	is a directed edge in $G_n$ from each word in this sequence to its follower,
	the sequence corresponds to an Hamiltonian path in $G_n$. The following theorem
	states that this path maps, by the standard correspondence described above, to
	the reverse of the prefer-map sequence:
	
	\begin{theorem} If we take the first letter of each word in $S$ we get the
		sequence described in Algorithm~\ref{inf-construction}. \end{theorem}
	
	This result allows for another construction of the reversed prefer-max
	sequence, as follows. By the definition of the necklace joining construction,
	shown in Algorithm~\ref{necklace-joining}, one can infer that key words appear
	in lexicographic order and that the distance between one key words to the next
	is the size of the necklace of the latter. For a word $w\in\N^n$, the
	\emph{primitive root} of $w$ is the  unique primitive (non-periodic) word
	$\rho(w)$ such that $w = \rho(w)^l$ for some $l \geq 1$. Since the length of
	the primitive root of a ket word is the length of its cycle, we get the
	following construction:
	
	
	\begin{algorithm}[!h] Construct an infinite sequence of symbols by
		concatenating the primitive roots of the key words of length $n$ in
		left-to-right lexicographic order. \caption{Word concatenation construction.}
		\label{word-concatenation} \end{algorithm}
	
	As the reader may expect, our result is this sequence is also the reverse of
	the prefer-max sequence:
	
	\begin{theorem} The sequence described in Algorithm~\ref{word-concatenation} is
		the same sequence described in Algorithm~\ref{inf-construction}. \end{theorem}
	
	The last result is similar in nature to the theorem of Fredricksen and Maiorana
	in~\cite{Fredricksen1978} (see also~\cite{Moreno2004} and~\cite{Moreno2015}).
	These papers show that the concatenation in lexicographic order of the Lyndon
	words (words that are minimal in lexicographic among their rotations) of length
	dividing $n$ produces a de Bruijn sequence of span $n$, and that this word is
	lexicographically minimal among all de Bruijn sequences of span $n$. In another
	paper~\cite{Fredricksen1970}, Fredricksen proves that (in the binary case) the
	prefer-max (there called Ford) sequence is the lexicographically least
	sequence. With some mapping (changing the order of the letters and the
	direction of reading words when considering lexicographic order) our last
	result can be regarded as an alternative proof of the result identified by
	Fredricksen and Maiorana.
	
	
	\section{Proofs}
	
	While Theorem~\ref{thm:onion} can be proved indirectly using the fact that the
	necklace joining construction, described in Algorithm~\ref{necklace-joining},
	gives the reverse of the pref-max construction, described in
	Algorithm~\ref{pref-max}, we bring here a direct proof that uses only the
	definition of the prefer-max sequence as the sequence generated by
	Algorithm~\ref{pref-max}. The main reason for this choice is because we view
	this observation as the starting point of our exploration so we do not want its
	proof to depend on the other results.
	
	
	\begin{repeatedtheorem}{thm:onion}[Onion] The $(k-1,n)$-prefer-max sequence is
		a suffix of the $(k,n)$-prefer-max sequence. \end{repeatedtheorem}
	\begin{proof} Denote the symbols of the $(n,k$)-prefer-max sequence by
		$s_1,s_2,\dots,s_{k^n+1}$ and let, for notational convenience, $s_i=0$ for all
		$i<1$.  For each $i\geq 1$ let $w_i=s_{i-n+1}\cdots s_i$ be the ``sliding
		window'' observed at the $i$th symbol of the sequence. The proof goes by
		showing that $\{w_1,\dots,w_{k^n-(k-1)^n}\} = \{0,\dots,k-1\}^n \setminus
		\{0,\dots,k-2\}^n$.
		
		Let ${i_0} = \min\{i\colon k-1 \notin w_i\}$. By minimality, because
		$s_{{i_0}-n}$ is the only symbol that is in $w_{i_0-1}$ and is not in
		$w_{i_0}$, we have that $s_{{i_0}-n}=k-1$. Since $s_{i_0} \neq k-1$ and
		because $s_{{i_0}-n+1}\cdots s_{{i_0}-1} (k-2) \notin \{w_1,\dots,w_{i_0-1}\}$
		because it does not contain the symbol $k-1$, we get, by the construction of
		the prefer-max sequence, that $s_{{i_0}}=k-2$. This means that there exists
		some ${i_1}<{i_0}$ such that $w_{i_1} = s_{{i_0}-n+1}\cdots s_{{i_0}-1}(k-1)$.
		
		Assume, towards contradiction, that $s_{{i_0}-n+1}\cdots s_{{i_0}-1} \neq
		0^{n-1}$. In particular $w_{i_1} \neq 0^{n-1}(k-1)$ which means that ${i_1} >
		1$. Therefore, $k-1  \in w_{i_1-1}=
		\{s_{{i_1}-n},\dots,s_{{i_1}-1}\}=\{s_{{i_1}-n}\} \cup
		\{s_{{i_0}-n+1},\dots,s_{{i_0}-1}\}$. Since the second set in this union does
		not contain $k-1$, we get that $s_{{i_1}-n}=k-1$. This leads to a
		contradiction because it means that $w_{{i_1}-1} = w_{{i_0}-1}$ which cannot
		happen in a de Bruijn sequence. This contradiction gives us that
		$s_{{i_0}-n+1}\cdots s_{{i_0}-1} = 0^{n-1}$.
		
		The sequence is therefore as follows:
		
		$$\underbrace{s_1,\dots,s_{i_0-n-1},k-1,\overbrace{0,\dots,0}^\text{$n\text{-}1$ times}}_{\text{all the $w_i$s here contain $k-1$}},\underbrace{k-2}_{s_{i_0}},s_{i_0+1},\dots,s_{k^n+1}$$ To complete the argument, we need to show that all substrings of length $n$ that contain $k-1$ appear before $s_{i_0}$.
		
		For $w \in \{0,\dots,k-1\}^n$ and $1 \leq m \leq n$, let $j_m$ be such that
		$w_{j_m} = w[m..n]0^{m-1}$, where $w[m..n]$ is the $n-m+1$ letters suffix of
		$w$. Because zero is the smallest symbol in our alphabet, we get that it is
		added by the prefer-max rule only if all the other symbols cannot be added,
		i.e., when adding other symbols would generate a substring that has already
		being seen before. Since $w$ comes, as a window, right before a word of the
		form $w[2..n]\tau$ for some $\tau \geq 0$, we have that $w[2..n]0$ comes after
		(or is equal to) $w[2..n]\tau$ which is after $w$. Similar arguments give us
		that $w[2..n]0$ appears before $w[3..n]0^2$ which appears before $w[4..n]0^3$
		and so on. Together we get that $j_1 < j_2 < \cdots < j_n$. Note that
		$w_{j_1}=w$, i.e, that $w$ appears as a substring at the $j_1$'s symbol of the
		sequence.
		
		We will now show that if $w$ contains the symbol $k-1$ it must appear as a
		substring before the $i_0$'s symbol in the sequence, i.e., that $j_1 < i_0$.
		Consider first the case where the last (right-most) symbol of $w$ is $k-1$. In
		this case, $j_n=i_{0}-1$ so we have, because $j_1 < j_n$, that $w$ appears as
		a substring before the $i_0$'th symbol in the sequence. The second case is
		when the last (right-most) symbol in $w$ is not $k-1$ but there is another
		letter in $w$ which is $k-1$. In this case we have that $w$ appears less than
		$n$ steps after a window that ends with $k-1$ and the window
		$w_{i_0-1}=(k-1)0^{n-1}$ does not appear in the windows between them because
		they all contain $k-1$ at a letter that is not  their first. Formally, we have
		that there is some $m<n$ such that $j_1-m < i_0$,  $w \in
		\{w_{{j_1}-m},\dots,w_{j_1-1}\}$ and $w_{i_0-1} \notin
		\{w_{{j_1}-m},\dots,w_{j_1-1}\}$. This gives us that, also in this case, the
		window $w$ appears before	the index $i_0$.
		
		To conclude the proof we note that we established that the windows up to the
		index $i_0$ contain all and only the words in $\{0,\dots,k-1\}^n \setminus
		\{0,\dots,k-2\}^n$ and that this part of the sequence ends with $n-1$ zeros
		and then the next part begins with the symbol $k-2$. This means that from this
		point on the prefer-max construction acts exactly as it does when the sequence
		begins with $k-2$ so it must constructs the $(k-1,n)$-prefer-max sequence.
	\end{proof}
	
	
	
	\bibliography{library}{} \bibliographystyle{plain}
	
\end{document}
